\documentclass[a4paper]{article}
\title{Direct collocation}
\author{Norbert Braun}
\begin{document}
\maketitle
\section{Approximation of $x$ and $\dot{x}$ at midpoint}
Let
\begin{equation}
y(s) = c_0 + c_1 s + c_2 s^2 + c_3 s^3
\end{equation}
Then,
\begin{equation}
\dot{y}(s) = c_1 + 2 c_2 s + 3 c_3 s^2
\end{equation}
Let $y(0)$, $\dot{y}(0)$, $y(1)$ and $\dot{y}(1)$ be given. This allows to determine $c_0,\ldots,c_3$.
We have
\begin{eqnarray}
y(0) = c_0 &\qquad& \dot{y}(0) = c_1\\
y(1) = c_0 + c_1 + c_2 + c_3 &\qquad& \dot{y}(1) = c_1 + 2c_2 + 3c_3 \nonumber
\end{eqnarray}
This can be inverted to give
\begin{eqnarray}
c_0 &=& y(0)\\
c_1 &=& \dot{y}(0)\nonumber \\
c_2 &=& -3y(0) - 2\dot{y}(0) + 3y(1) - \dot{y}(1)\nonumber \\
c_3 &=& 2y(0) + \dot{y}(0) - 2y(1) + \dot{y}(1)\nonumber
\end{eqnarray}
We can thus obtain an approximation of $y(\frac{1}{2})$ and $\dot{y}(\frac{1}{2})$.
\begin{eqnarray}
y_c &:=& y(\frac{1}{2})\\
&=& c_0 + \frac{1}{2} c_1 + \frac{1}{4} c_2 + \frac{1}{8} c_3\nonumber \\
&=& \frac{1}{2} \left( y(0) + y(1) \right) + \frac{1}{8} \left( \dot{y}(0) - \dot{y}(1) \right)\nonumber
\end{eqnarray}
and
\begin{eqnarray}
\dot{y}_c &:=& \dot{y}(\frac{1}{2})\\
&=& c_1 + c_2 + \frac{3}{4} c_3\nonumber \\
&=& -\frac{3}{2} \left( y(0) - y(1) \right) - \frac{1}{4} \left( \dot{y}(0) + \dot{y}(1) \right) \nonumber
\end{eqnarray}
Finally, let $x(t) = y(\frac{t}{h})$. Then $\dot{x} = \frac{1}{h} \dot{y}$. Let $x_0 := x(0)$, $x_1 := x(h)$, and $x_c := x(\frac{h}{2}) = y(\frac{1}{2}) = y_c$. Thus
\begin{eqnarray}
x_c &=& \frac{1}{2}\left( x_0 + x_1 \right) + \frac{h}{8} \left( \dot{x}_0 - \dot{x}_1 \right)\\
\dot{x}_c &=& - \frac{3}{2h} \left( x_0 - x_1 \right) - \frac{1}{4} \left( \dot{x}_0 + \dot{x}_1 \right)
\end{eqnarray}

\section{Solving ODEs}
Consider a dynamical system with $n$ degrees of freedom and $m$ control inputs. This gives rise to an $n$-dimensional second-order ODE,
\begin{equation}
\ddot{q}_k = f_k(q_1, \ldots, q_n, \dot{q}_1, \ldots, \dot{q}_n, u_1, \ldots, u_m)
\qquad k = 1,\ldots,n
\end{equation}
As usual, we turn this in a $2n$-dimensional first-order ODE by introducing $n$ dummy variables $v_k$:
\begin{eqnarray}
\dot{q}_k &=& v_k\\
\dot{v}_k &=& f_k(q_1, \ldots, q_n, v_1, \ldots, v_n, u_1, \ldots, u_m)\nonumber
\end{eqnarray}

For the direct collocation method, we divide our time span of interest into $N$ sub-intervals. For each interval, we can use the equations from above to obtain values and derivatives at the interval midpoint.
\begin{eqnarray}
q_k^{(c)} &=& \frac{1}{2} \left( q_k^{(i)} + q_k^{(i+1)} \right) +
    \frac{h}{8} \left( v_k^{(i)} - v_k^{(i+1)} \right)\\
\dot{q}_k^{(c)} &=& -\frac{3}{2h} \left( q_k^{(i)} - q_k^{(i+1)} \right) -
    \frac{1}{4} \left( v_k^{(i)} + v_k^{(i+1)} \right)\\
v_k^{(c)} &=& \frac{1}{2} \left( v_k^{(i)} + v_k^{(i+1)} \right)\\
    &&+ \frac{h}{8} \left( f_k(q^{(i)}, v^{(i)}, u^{(i)}) - f_k(q^{(i+1)}, v^{(i+1)}, u^{(i+1)}) \right)
\nonumber \\
\dot{v}_k^{(c)} &=& - \frac{3}{2h} \left( v_k^{(i)} - v_k^{(i+1)} \right)\\
    &&- \frac{1}{4} \left( f_k(q^{(i)}, v^{(i)}, u^{(i)}) + f_k(q^{(i+1)}, v^{(i+1)}, u^{(i+1)}) \right)
\nonumber
\end{eqnarray}
In these equations, $k = 1,\ldots,n$ is a degree of freedom index and $i = 0,\ldots,N-1$ is a node index.

For the controls, we use a linear approximation,
\begin{equation}
u_k^{(c)} = \frac{1}{2} \left( u_k^{(i)} + u_k^{(i+1)} \right)
\end{equation}

We solve the ODE by enforcing it at the interval midpoints,
\begin{eqnarray}
0 &\stackrel{!}{=}& g_k^{(i),0} = v_k^{(c)} - \dot{q}_k^{(i)}\\
&=& \frac{3}{2h} \left( q_k^{(i)} - q_k^{(i+1)} \right)
+ \frac{3}{4}\left( v_k^{(i)} + v_k^{(i+1)} \right)\nonumber \\
&&+ \frac{h}{8}\left( f_k(q^{(i)}, v^{(i)}, u^{(i)}) - f_k(q^{(i+1)}, v^{(i+1)}, u^{(i+1)}) \right)\nonumber
\end{eqnarray}
\begin{eqnarray}
0 &\stackrel{!}{=}& g_k^{(i),1} = f_k^{(c)} - \dot{v}_k^{(c)}\\
&=& f_k(q^{(c)}, \dot{q}^{(c)}, u^{(c)}) + \frac{3}{2h} \left( v_k^{(i)} - v_k^{(i+1)} \right) \nonumber \\
&& + \frac{1}{4} \left( f_k(q^{(i)}, v^{(i)}, u^{(i)}) + f_k(q^{(i+1)}, v^{(i+1)}, u^{(i+1)}) \right) \nonumber\\
&=:& h_k(q^{(i)}, v^{(i)}, u^{(i)}, q^{(i+1)}, v^{(i+1)}, u^{(i+1)}) + \frac{3}{2h} \left( v_k^{(i)} - v_k^{(i+1)} \right) \nonumber \\
&& + \frac{1}{4} \left( f_k(q^{(i)}, v^{(i)}, u^{(i)}) + f_k(q^{(i+1)}, v^{(i+1)}, u^{(i+1)}) \right) \nonumber
\end{eqnarray}

The arguments of $f_k$ in the equation above turn out to be a linear function in the arguments of $h_k$,
\begin{equation}
\left( \begin{array}{c} q_k^{(c)}\\ \dot{q}_k^{(c)}\\ u_k^{(c)} \end{array} \right) =
\left( \begin{array}{cccccc}
\frac{1}{2} & \frac{h}{8} & 0 & \frac{1}{2} & -\frac{h}{8} & 0\\
-\frac{3}{2h} & -\frac{1}{4} & 0 & \frac{3}{2h} & -\frac{1}{4} & 0\\
0 & 0 & \frac{1}{2} & 0 & 0 & \frac{1}{2}
\end{array} \right)
\left( \begin{array}{c}
q_k^{(i)}\\ v_k^{(i)}\\ u_k^{(i)}\\ q_k^{(i+1)}\\ v_k^{(i+1)}\\ u_k^{(i+1)}
\end{array} \right)
\end{equation}

\section{Cost function}
We consider a standard cost functional $L$ of the form
\begin{eqnarray}
L &=& \frac{1}{2} \int_0^T\mathrm{d}t\;\left[
\left(\begin{array}{cc}q(t)\\ \dot{q}(t)\end{array}\right)^\top
Q
\left(\begin{array}{cc}q(t)\\ \dot{q}(t)\end{array}\right)
+ u(t)^\top R u(t)
\right]\\
\nonumber &=:& \int_0^T\mathrm{d}t\; l(q(t), \dot{q}(t), u(t))
\end{eqnarray}

Let
\begin{equation}
w(t) = \int_0^t \mathrm{d}s\; l(q(s), \dot{q}(s), u(s))
\end{equation}
or equivalently,
\begin{equation}
\label{eqn:diff-w}
w(0) = 0,\;\dot{w}(t) = l(q(t), \dot{q}(t), u(t))
\end{equation}
Our optimization objective is to minimize $L = w(T)$ ($w_N$ after discretization).

As above, $w$ at interval midpoint is given by
\begin{equation}
\dot{w}^{(c)} = - \frac{3}{2h} (w^{(i)} - w^{(i+1)}) - \frac{1}{4}(l^{(i)} + l^{(i+1)})
\end{equation}
so enforcing (\ref{eqn:diff-w}) leads to the constraints
\begin{equation}
0 \stackrel{!}{=} g^{(i),2} = l^{(c)} - \dot{w}^{(c)}
\end{equation}
plus a trivial boundary constraint.

\subsection{Derivatives of quadratic forms}
Let
\begin{equation}
f(x) = \frac{1}{2} x^\top Q x
\end{equation}
where we assume, without loss of generality, that $Q$ is symmetric: $Q^\top = Q$.

The derivatives of $f$ are then given by
\begin{eqnarray}
\frac{\mathrm{d}f}{\mathrm{d}x} &=& Q x\\
\frac{\mathrm{d}^2 f}{\mathrm{d}x^2} &=& Q
\end{eqnarray}
Equivalently, in index form,
\begin{eqnarray}
f(x_1,\ldots,x_n) &=& \frac{1}{2} \sum_{i,j} Q_{ij} x_i x_j\\
\frac{\mathrm{d}f}{\mathrm{d}x_a} &=& \sum_{i} Q_{ai} x_i\\
\frac{\mathrm{d}^2 f}{\mathrm{d}x_a \mathrm{d}x_b} &=& Q_{ab}
\end{eqnarray}

\section{Solving the constrained optimization problem}
We can now solve the constrained optimization problem, e.g. with \textsc{Ipopt}. There are $(2n + m + 1)\cdot (N+1)$ optimization variables:
\begin{eqnarray*}
q_0^{(0)}, \ldots, q_n^{(0)}, \ldots, q_0^{(N)}, \ldots, q_n^{(N)},\\
v_0^{(0)}, \ldots, v_n^{(0)}, \ldots, v_0^{(N)}, \ldots, v_n^{(N)},\\
u_0^{(0)}, \ldots, u_m^{(0)}, \ldots, u_0^{(N)}, \ldots, u_m^{(N)},\\
w^{(0)}, \ldots, w^{(N)}
\end{eqnarray*}

We write $\left. \frac{\partial f_k}{\partial q_l} \right|_{(i)}$ as a shorthand for $\left. \frac{\partial f_k}{\partial v_l} \right|_{(q^{(i)}, v^{(i)}, u^{(i)})}$.

\subsection{Derivatives}
\begin{eqnarray}
\frac{\partial g_k^{(i),0}}{\partial q_l^{(i)}} 
&=& \left. \frac{h}{8} \frac{\partial f_k}{\partial q_l} \right|_{(i)}
 + \frac{3}{2h} \delta_{k,l}\\
\frac{\partial g_k^{(i),0}}{\partial q_l^{(i+1)}} 
&=& -\left. \frac{h}{8} \frac{\partial f_k}{\partial q_l} \right|_{(i+1)}
 - \frac{3}{2h} \delta_{k,l}\\
\frac{\partial g_k^{(i),0}}{\partial v_l^{(i)}} 
&=& \left. \frac{h}{8} \frac{\partial f_k}{\partial v_l} \right|_{(i)}
 + \frac{3}{4} \delta_{k,l}\\
\frac{\partial g_k^{(i),0}}{\partial v_l^{(i+1)}} 
&=& -\left. \frac{h}{8} \frac{\partial f_k}{\partial v_l} \right|_{(i+1)}
 + \frac{3}{4} \delta_{k,l}\\
 \frac{\partial g_k^{(i),0}}{\partial u_p^{(i)}} 
&=& \left. \frac{h}{8} \frac{\partial f_k}{\partial u_p} \right|_{(i)}\\
 \frac{\partial g_k^{(i),0}}{\partial u_p^{(i+1)}} 
&=& -\left. \frac{h}{8} \frac{\partial f_k}{\partial u_p} \right|_{(i+1)}
\end{eqnarray}

\begin{eqnarray}
\frac{\partial g_k^{(i),1}}{\partial q_l^{(i)}}
&=& \left. \frac{\partial h_k}{\partial q_l^{(i)}} \right|_{(i), (i+1)}
 + \frac{1}{4} \left. \frac{\partial f_k}{\partial q_l} \right|_{(i)}\\
&=& \frac{1}{2} \left. \frac{\partial f_k}{\partial q_l} \right|_{(c)}
 - \frac{3}{2h} \left. \frac{\partial f_k}{\partial \dot{q}_l} \right|_{(c)}
 + \frac{1}{4} \left. \frac{\partial f_k}{\partial q_l} \right|_{(i)}
\nonumber\\
\frac{\partial g_k^{(i),1}}{\partial q_l^{(i+1)}}
&=& \left. \frac{\partial h_k}{\partial q_l^{(i+1)}} \right|_{(i), (i+1)}
 + \frac{1}{4} \left. \frac{\partial f_k}{\partial q_l} \right|_{(i+1)}\\
 &=& \frac{1}{2} \left. \frac{\partial f_k}{\partial q_l} \right|_{(c)}
 + \frac{3}{2h} \left. \frac{\partial f_k}{\partial \dot{q}_l} \right|_{(c)}
 + \frac{1}{4} \left. \frac{\partial f_k}{\partial q_l} \right|_{(i+1)}
\nonumber\\
\frac{\partial g_k^{(i),1}}{\partial v_l^{(i)}}
&=& \left. \frac{\partial h_k}{\partial v_l^{(i)}} \right|_{(i), (i+1)}
 + \frac{1}{4} \left. \frac{\partial f_k}{\partial v_l} \right|_{(i)}
 + \frac{3}{2h} \delta_{k,l}\\
&=& \frac{h}{8} \left. \frac{\partial f_k}{\partial q_l} \right|_{(c)}
 - \frac{1}{4} \left. \frac{\partial f_k}{\partial \dot{q}_l} \right|_{(c)}
 + \frac{1}{4} \left. \frac{\partial f_k}{\partial v_l} \right|_{(i)}
 + \frac{3}{2h} \delta_{k,l}
\nonumber\\
\frac{\partial g_k^{(i),1}}{\partial v_l^{(i+1)}}
&=& \left. \frac{\partial h_k}{\partial v_l^{(i+1)}} \right|_{(i), (i+1)}
 + \frac{1}{4} \left. \frac{\partial f_k}{\partial v_l} \right|_{(i+1)}
 - \frac{3}{2h} \delta_{k,l}\\
&=& - \frac{h}{8} \left. \frac{\partial f_k}{\partial q_l} \right|_{(c)}
 - \frac{1}{4} \left. \frac{\partial f_k}{\partial \dot{q}_l} \right|_{(c)}
 + \frac{1}{4} \left. \frac{\partial f_k}{\partial v_l} \right|_{(i+1)}
 - \frac{3}{2h} \delta_{k,l}
\nonumber\\
\frac{\partial g_k^{(i),1}}{\partial u_p^{(i)}}
&=& \left. \frac{\partial h_k}{\partial u_p^{(i)}} \right|_{(i), (i+1)}
 + \frac{1}{4} \left. \frac{\partial f_k}{\partial u_p} \right|_{(i)}\\
&=& \frac{1}{2} \left. \frac{\partial f_k}{\partial u_p} \right|_{(c)}
 + \frac{1}{4} \left. \frac{\partial f_k}{\partial u_p} \right|_{(i)}
\nonumber\\
\frac{\partial g_k^{(i),1}}{\partial u_p^{(i+1)}}
&=& \left. \frac{\partial h_k}{\partial u_p^{(i+1)}} \right|_{(i), (i+1)}
 + \frac{1}{4} \left. \frac{\partial f_k}{\partial u_p}\right|_{(i+1)}\\
&=& \frac{1}{2} \left. \frac{\partial f_k}{\partial u_p} \right|_{(c)}
 + \frac{1}{4} \left. \frac{\partial f_k}{\partial u_p}\right|_{(i+1)}
\nonumber
\end{eqnarray}

\subsection{Second derivatives}
We let $y_l^{(i)}$ stand as a placeholder for $q_l^{(i)}$, $v_l^{(i)}$, $u_l^{(i)}$.
\begin{eqnarray}
\frac{\partial^2 g_k^{(i),0}}{\partial y_l^{(i)} \partial {y^\prime}_p^{(i)}} &=&
\frac{h}{8} \frac{\partial^2 f_k}{\partial y_l^{(i)} \partial {y^\prime}_p^{(i)}}\\
\frac{\partial^2 g_k^{(i),0}}{\partial y_l^{(i+1)} \partial {y^\prime}_p^{(i+1)}} &=&
-\frac{h}{8} \frac{\partial^2 f_k}{\partial y_l^{(i+1)} \partial {y^\prime}_p^{(i+1)}}\\
\frac{\partial^2 g_k^{(i),0}}{\partial y_l^{(i)} \partial {y^\prime}_p^{(i+1)}} &=& 0
\end{eqnarray}
\begin{eqnarray}
\frac{\partial^2 g_k^{(i),1}}{\partial y_l^{(i)} \partial {y^\prime}_l^{(i)}} &=&
\left. \frac{\partial^2 h_k}{\partial y_l^{(i)} \partial {y^\prime}_l^{(i)}} \right|_{(i), (i+1)} +
\frac{1}{4} \frac{\partial^2 f_k}{\partial y_l^{(i)} \partial {y^\prime}_l^{(i)}}\\
\frac{\partial^2 g_k^{(i),1}}{\partial y_l^{(i+1)} \partial {y^\prime}_l^{(i+1)}} &=&
\left. \frac{\partial^2 h_k}{\partial y_l^{(i+1)} \partial {y^\prime}_l^{(i+1)}} \right|_{(i), (i+1)} +
\frac{1}{4} \frac{\partial^2 f_k}{\partial y_l^{(i+1)} \partial {y^\prime}_l^{(i+1)}}\\
\frac{\partial^2 g_k^{(i),1}}{\partial y_l^{(i)} \partial {y^\prime}_l^{(i+1)}} &=&
\left. \frac{\partial^2 h_k}{\partial y_l^{(i+1)} \partial {y^\prime}_l^{(i+1)}} \right|_{(i), (i+1)}
\end{eqnarray}

\subsection{Sparsity structure of the Hessian of the Lagrangian}
For \textsc{Ipopt}, the Hessian of the Lagrangian is
\begin{equation}
\sigma_f \nabla^2 f(x_k) + \sum_{i=1}^{m} \lambda_i \nabla^2 g_i(x_k)
\end{equation}

The objective is simply equal to the optimization variable $w_N$, and $\nabla^2 f(x_k)$ is thus zero. The constraints link two neighbouring time nodes. Thus, the Hessian of the Lagrangian has on-diagonal blocks, involving variables at some node $i$, and off-diagonal blocks, involving variables from nodes $i$ and $i+1$. All other blocks are zero.
\end{document}
