\documentclass[a4paper]{article}
\usepackage{amsmath}
\usepackage{bbold}
\title{The center of pressure (CoP)}
\author{Norbert Braun}
\begin{document}
\maketitle
Consider a rigid body in contact with the ground plane. Let the ground plane normal be $n$. Let various contact forces $F_i$ act on the rigid body at positions $r_i$. The center of pressure is then defined as
\begin{equation}
x_c = \frac{\sum_i (F_i \cdot n) r_i}{\sum_i F_i \cdot n}
\end{equation}
where the generalization from finitely many forces to a force field should be obvious. As $F_i \cdot n \ge 0$, the center of pressure must lie within the convex hull of the contact points $r_i$.

Usually, we are given the total force and the total torque on the rigid body,
\begin{eqnarray}
F &=& \sum_i F_i\\
T &=& \sum_i r_i \times F_i
\end{eqnarray}
To calculate the center of pressure from these vectors, we note that
\begin{eqnarray}
\nonumber n \times T &=& n \times \left( \sum_i r_i \times F_i \right) = \sum_i (n \times (r_i \times F_i))\\
&=& \sum_i \left( r_i (n \cdot F_i) - F_i (n \cdot r_i) \right)
\end{eqnarray}
As the contact points all lie in the plane, $n \cdot r_i$ is equal for all $i$. Let $n \cdot r_i =: h$. Then
\begin{equation}
n \times T = \sum_i r_i (n \cdot F_i) - h \sum_i F_i = \sum_i r_i (n \cdot F_i) - h F
\end{equation}
Thus
\begin{equation}
x_c = \frac{n \times T + h F}{F \cdot n}
\end{equation}
\end{document}
