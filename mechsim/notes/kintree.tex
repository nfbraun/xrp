\documentclass[a4paper]{article}
\usepackage{amsmath}
\usepackage{bbold}
\title{Lagrangian mechanics for kinematic trees}
\author{Norbert Braun}
\begin{document}
\maketitle
\section{Affine transforms}
An \textit{affine transform} is a transform $x^\prime \to x$ of the form
\begin{equation}
x^\prime = Rx + c
\end{equation}
for $x, x^\prime \in \mathbb{R}^n$ (typically $\mathbb{R}^3$), where $c \in \mathbb{R}^n$ and $R \in \mathbb{R}^{n \times n}$ with $R^\top R = \mathbb{1}$.

Let us consider \textit{augmented vectors} $\overline{x} \in \mathbb{R}^{n+1}$ of the form
\begin{equation}
\overline{x} = (x, 1)^\top = (x_1, \ldots, x_n, 1)^\top
\end{equation}
and matrices $T \in \mathbb{R}^{(n+1)\times(n+1)}$ of the form
\begin{equation}
T := \left( \begin{array}{cc}
R & c\\
0 & 1\end{array}\right)
\end{equation}
Then, for $\overline{x} = T \overline{x}^\prime$, $x = R x^\prime + c$.

The inverse of $T$ is given by
\begin{equation}
T^{-1} = \left(\begin{array}{cc}
R^\top & -R^\top c\\
0 & 1
\end{array}\right)
\end{equation}

\section{Kinematic trees}
We assume the existance of an inertial (``fixed'') frame, called the world frame. Let vectors in the world frame be denoted by $x$.

For each body, we define a body frame. Let vectors in the body frame be denoted by $x^\prime$ (we rely on context to distinguish between the frames of the different bodies). $x$ and $x^\prime$ are related by an affine transform,
\begin{equation}
\overline{x} = T \overline{x}^\prime
\end{equation}

\section{Kinetic energy}
Consider a point mass $m_i$ at position $x^\prime_i$ (in body coordinates) inside the body. The body is assumed to be rigid, i.e.
\begin{equation}
\dot{x^\prime_i} = 0
\end{equation}
and the velocity $\dot{x}_i$ of the point mass in world coordinates is thus given by
\begin{equation}
\dot{\overline{x}}_i = \dot{T} \overline{x}^\prime_i
\end{equation}
where $\dot{\overline{x}}_i = (\dot{x}_i, 0)^\top$, and thus $\dot{x}_i^\top \dot{x}_i = \dot{\overline{x}}_i^\top \dot{\overline{x}}_i$.

For the kinetic energy associated with the motion of the point mass, we have
\begin{equation}
E_{\mathrm{kin},i} = \frac{1}{2} m_i \dot{\overline{x}}_i^\top \dot{\overline{x}}_i
= \frac{1}{2} m_i \operatorname{tr} (\dot{\overline{x}}_i \dot{\overline{x}}_i^\top )
= \frac{1}{2} m_i \operatorname{tr} (\dot{T} \overline{x}^\prime_i {\overline{x}^\prime_i}^\top \dot{T}^\top )
\end{equation}

Now, let the body consist of many point masses. The kinetic energy is then
\begin{equation}
E_\mathrm{kin} = \sum_i E_{\mathrm{kin},i}
= \frac{1}{2} \sum_i m_i \operatorname{tr} (\dot{T} \overline{x}^\prime_i {\overline{x}^\prime_i}^\top \dot{T}^\top )
\end{equation}
Making use of the linearity of the trace, we rewrite this as
\begin{equation}
\label{eqn:ekindef}
E_\mathrm{kin} = \frac{1}{2}\operatorname{tr} \left( \dot{T} \left(\sum_i m_i \overline{x}^\prime_i {\overline{x}_i^\prime}^\top \right) \dot{T}^\top \right)
:= \frac{1}{2} \operatorname{tr} (\dot{T} J \dot{T}^\top )
\end{equation}
where
\begin{equation}
J = \sum_i m_i \overline{x}_i^\prime {\overline{x}_i^\prime}^\top
\end{equation}

This naturally generalizes to the case of mass distributions:
\begin{equation}
J = \int \operatorname{d}^3\!x^\prime\; \rho(x^\prime) \overline{x}^\prime\,{\overline{x}^\prime}^\top
\end{equation}

Usually, we are not given the mass distribution $\rho(x^\prime)$, but rather the total mass
\begin{equation}
m_\mathrm{tot} := \int \operatorname{d}^3\!x^\prime\; \rho(x^\prime)
\end{equation}
the center of gravity
\begin{equation}
\label{eqn:xpcdef}
x_c^\prime := \frac{1}{m_\mathrm{tot}} \int \operatorname{d}^3\!x^\prime\; \rho(x^\prime) x^\prime
\end{equation}
and the moment of inertia about the center of gravity,
\begin{equation}
I_c := \int \operatorname{d}^3\!x^\prime\; \rho(x^\prime) \left( ( (x^\prime - x_c^\prime)^\top (x^\prime - x_c^\prime) ) \mathbb{1} - (x^\prime - x_c^\prime)(x^\prime - x_c^\prime)^\top \right)
\end{equation}

To continue, we note that
\begin{eqnarray}
\nonumber \int \operatorname{d}^3\!x^\prime\; \rho(x^\prime) (x^\prime - x_c^\prime)
&=& \int \operatorname{d}^3\!x^\prime\; \rho(x^\prime) x^\prime
- x_c^\prime \int \operatorname{d}^3\!x^\prime\; \rho(x^\prime)\\
&=& m_\mathrm{tot} x_c^\prime - x_c^\prime m_\mathrm{tot} = 0
\end{eqnarray}
which implies, for arbitrary (constant) $a \in \mathbb{R}^3$,
\begin{equation}
\int \operatorname{d}^3\!x^\prime\; \rho(x^\prime) a^\top (x^\prime - x_c^\prime) = 0
\end{equation}
and
\begin{equation}
\int \operatorname{d}^3\!x^\prime\; \rho(x^\prime) a (x^\prime - x_c^\prime)^\top = 0
\end{equation}
(note that the first 0 is a scalar zero, and the second 0 is a zero matrix).

Now we can calculate the moment of inertia about the body frame origin,
\begin{eqnarray}
\label{eqn:i}
\nonumber I &:=& \int \operatorname{d}^3\!x^\prime\; \rho(x^\prime) \left( ({x^\prime}^\top x^\prime) \mathbb{1} - x^\prime {x^\prime}^\top \right)\\
\nonumber&=& \int \operatorname{d}^3\!x^\prime\; \rho(x^\prime) \left( ((x^\prime - x_c^\prime + x_c^\prime)^\top (x^\prime - x_c^\prime + x_c^\prime)) \mathbb{1} - (x^\prime - x_c^\prime + x_c^\prime)(x^\prime - x_c^\prime + x_c^\prime)^\top \right)\\
\nonumber&=& \int \operatorname{d}^3\!x^\prime\; \rho(x^\prime) \left( ((x^\prime - x_c^\prime)^\top (x^\prime - x_c^\prime)) \mathbb{1} - (x^\prime - x_c^\prime)(x^\prime - x_c^\prime)^\top \right.\\
\nonumber && \left. + 2 ({x_c^\prime}^\top (x^\prime - x_c^\prime)) \mathbb{1} - 2 x_c^\prime (x^\prime - x_c^\prime)^\top + ({x_c^\prime}^\top x_c^\prime) \mathbb{1} - x_c^\prime {x_c^\prime}^\top \right)\\
&=& I_c + m_\mathrm{tot}({x_c^\prime}^\top x_c^\prime) \mathbb{1} - m_\mathrm{tot}x_c^\prime {x_c^\prime}^\top
\end{eqnarray}

Let us now examine $J = \int \operatorname{d}^3\!x^\prime\; \rho(x^\prime) \overline{x}^\prime\,{\overline{x}^\prime}^\top$ in more detail. Keeping in mind that $\mathbb{R}^4 \ni \overline{x}^\prime = (x^\prime, 1)$, we have that
\begin{equation}
\overline{x}^\prime {\overline{x}^\prime}^\top =
\left(\begin{array}{cc}
x^\prime {x^\prime}^\top & x^\prime\\
{x^\prime}^\top & 1
\end{array}\right)
\end{equation}
and therefore
\begin{equation}
\label{eqn:j}
J = \left(\begin{array}{cc}
\tilde{J} & x_c^\prime\\
{x_c^\prime}^\top & m_\mathrm{tot}
\end{array}\right)
\end{equation}
where
\begin{eqnarray}
\nonumber \tilde{J} &=& \int \operatorname{d}^3\!x^\prime\; \rho(x^\prime) x^\prime\,{x^\prime}^\top\\
\nonumber &=& \int \operatorname{d}^3\!x^\prime\; \rho(x^\prime) \left(x^\prime\,{x^\prime}^\top - ({x^\prime}^\top x^\prime) \mathbb{1} \right) + \left( \int \operatorname{d}^3\!x^\prime\; \rho(x^\prime) ({x^\prime}^\top{x^\prime}) \right) \mathbb{1}\\
&=& -I + \left( \int \operatorname{d}^3\!x^\prime\; \rho(x^\prime) ({x^\prime}^\top{x^\prime}) \right) \mathbb{1}
\end{eqnarray}
For the last step, we calculate
\begin{eqnarray}
\nonumber \operatorname{tr}((x^\top x) \mathbb{1} - x x^\top)
&=& x^\top x \operatorname{tr}(\mathbb{1}) - \operatorname{tr}(xx^\top)
= x^\top x \operatorname{tr}(\mathbb{1}) - x^\top x\\
&=& (\operatorname{tr}(\mathbb{1}) - 1) x^\top x
= 2 x^\top x
\end{eqnarray}
Due to the linearity of the trace, we conclude that
\begin{equation}
\int\operatorname{d}^3\!x^\prime\; \rho(x^\prime) {x^\prime}^\top x^\prime
= \frac{1}{2} \operatorname{tr} \left( \int\operatorname{d}^3\!x^\prime\; \rho(x^\prime) \left( ({x^\prime}^\top x^\prime) \mathbb{1} - x^\prime {x^\prime}^\top \right) \right)
\end{equation}
which gives
\begin{equation}
\label{eqn:jtilde}
\tilde{J} = -I + \frac{1}{2} \operatorname{tr}(I) \mathbb{1}
\end{equation}
Combining (\ref{eqn:jtilde}) and (\ref{eqn:i}), we get
\begin{eqnarray}
\label{eqn:jtilde2}
\nonumber \tilde{J} &=& -I_c - m_\mathrm{tot}({x_c^\prime}^\top x_c^\prime) \mathbb{1} + m_\mathrm{tot} x_c^\prime {x_c^\prime}^\top
+ \frac{1}{2} \operatorname{tr}\left(I_c + m_\mathrm{tot} ({x_c^\prime}^\top x_c^\prime) \mathbb{1} - m_\mathrm{tot} x_c^\prime {x_c^\prime}^\top\right)\\
\nonumber &=& -I_c - m_\mathrm{tot} ({x_c^\prime}^\top x_c^\prime) \mathbb{1} + m_\mathrm{tot} x_c^\prime {x_c^\prime}^\top + \frac{1}{2} \operatorname{tr}\left(I_c\right) \mathbb{1} + \frac{1}{2} m_\mathrm{tot} \left({x_c^\prime}^\top x_c^\prime (\operatorname{tr}(\mathbb{1}) - 1)\right) \mathbb{1}\\
&=& -I_c + m_\mathrm{tot} x_c^\prime {x_c^\prime}^\top + \frac{1}{2} \operatorname{tr}\left(I_c\right) \mathbb{1}
\end{eqnarray}
By (\ref{eqn:j}) and (\ref{eqn:jtilde2}), we have thus succeeded in expressing $J$ in terms of $m_\mathrm{tot}$, $x_c^\prime$ and $I_c$.

\section{Potential energy}
We restrict our analysis to the case of a homogeneous gravitational field, where the potential energy of a point mass at position $x_i$ (in world coordinates!) is given by
\begin{equation}
E_{\mathrm{pot},i} = - m_i g^\top x_i
\end{equation}

We define
\begin{equation}
\overline{g} := (g, 0)
\end{equation}
(which makes sense, because $g$ is an \emph{acceleration}, not a position). Then
\begin{equation}
E_{\mathrm{pot},i} = - m_i \overline{g}^\top \overline{x}_i
= - m_i \overline{g}^\top T \overline{x}^\prime_i
\end{equation}
where $T$ is the affine transform associated with the body frame, as above. In the case of a mass distribution, we get
\begin{equation}
E_\mathrm{pot} = -\int\operatorname{d}^3\!x^\prime\; \rho(x^\prime)\, \overline{g}^\top T \overline{x}^\prime
= -\overline{g}^\top T \int\operatorname{d}^3\!x^\prime\; \rho(x^\prime)\, \overline{x}^\prime
\end{equation}
With $\overline{x}_c^\prime := (x_c^\prime, 1)$ and (\ref{eqn:xpcdef}), we can rewrite this as
\begin{equation}
\label{eqn:epot}
E_\mathrm{pot} = - m_\mathrm{tot} \overline{g}^\top T \overline{x}_c^\prime
\end{equation}
so we have succeeded in expressing $E_\mathrm{pot}$ in terms of $m_\mathrm{tot}$ and $x_c^\prime$.

\section{The Lagrange equation}
We now need to consider the case of many rigid bodies. Therefore, we let the affine transform associated with the body frame of body $i$ be denoted by $T_i$.

Let the affine transforms be parameterized by generalized coordinates $q_j$. The time derivative of $T_i$ is then given by
\begin{equation}
\dot{T}_i = \sum_{j} \frac{\partial T}{\partial q_j} \dot{q}_j
\end{equation}

By (\ref{eqn:ekindef}), the kinetic energy of the rigid body is given by
\begin{equation}
E_{\mathrm{kin},i} = \frac{1}{2} \operatorname{tr} (\dot{T}_i J_i \dot{T}_i^\top)
\end{equation}
(note that the index now refers to a rigid body, no longer to a point mass inside a rigid body). This can be written in terms of the generalized coordinates as
\begin{equation}
E_{\mathrm{kin},i} = \frac{1}{2} \operatorname{tr} (\sum_{jk} \frac{\partial T_i}{\partial q_j} \dot{q}_j J_i \frac{\partial T_i^\top}{\partial q_k} \dot{q}_k)
= \frac{1}{2} \sum_{jk} \operatorname{tr} (\frac{\partial T_i}{\partial q_j} J_i \frac{\partial T_i^\top}{\partial q_k}) \dot{q}_j \dot{q}_k
\end{equation}

The kinetic energy of the whole kinematic tree is then given by
\begin{equation}
E_\mathrm{kin} = \sum_i E_{\mathrm{kin},i} = \frac{1}{2} \sum_{ijk} \operatorname{tr} (\frac{\partial T_i}{\partial q_j} J_i \frac{\partial T_i^\top}{\partial q_k}) \dot{q}_j \dot{q}_k
\end{equation}

This is a quadratic form in the generalized velocities $\dot{q}$. We write it as
\begin{equation}
E_\mathrm{kin} = \frac{1}{2} \dot{q}^\top M(q) \dot{q}
\end{equation}
where the mass matrix $M(q)$ is given by
\begin{equation}
M_{jk}(q) = \sum_i \operatorname{tr} (\frac{\partial T_i}{\partial q_j} J_i \frac{\partial T_i^\top}{\partial q_k})
\end{equation}

By (\ref{eqn:epot}), the potential energy for a single body is given by
\begin{equation}
E_{\mathrm{pot},i} = - m_{\mathrm{tot},i} \overline{g}^\top T_i \overline{x}_{c,i}^\prime
\end{equation}
so the total potential energy is
\begin{equation}
E_\mathrm{pot} = \sum_i E_{\mathrm{pot},i} = - \sum_i m_{\mathrm{tot},i} \overline{g}^\top T_i \overline{x}_{c,i}^\prime
\end{equation}
and the derivative that appears in the Lagrange equation is
\begin{equation}
\frac{\partial E_\mathrm{pot}}{\partial q_j} = - \sum_i m_{\mathrm{tot},i} \overline{g}^\top \frac{\partial T_i}{\partial q_j} \overline{x}_{c,i}^\prime
\end{equation}
\end{document}
