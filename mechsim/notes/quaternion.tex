\documentclass[a4paper]{article}
\usepackage{amsmath}
\usepackage{bbold}
\title{Quaternions}
\author{}
\date{}
\begin{document}
\maketitle
Let $q = (a,b,c,d) = a + b\mathbf{i} + c\mathbf{j} + d\mathbf{k} \in \mathbb{H}$.
\begin{itemize}
\item\textbf{Multiplication table:} From $\mathbf{i}^2 = \mathbf{j}^2 = \mathbf{k}^2 = \mathbf{i}\mathbf{j}\mathbf{k} = -1$, the whole multiplication table follows:

\begin{tabular}{l@{$\;=\;$}ll@{$\;=\;$}ll@{$\;=\;$}l}
$\mathbf{i}\mathbf{i}$ & $-1$ &
$\mathbf{j}\mathbf{i}$ & $-\mathbf{k}$ &
$\mathbf{k}\mathbf{i}$ & $\mathbf{j}$\\
$\mathbf{i}\mathbf{j}$ & $\mathbf{k}$ &
$\mathbf{j}\mathbf{j}$ & $-1$ &
$\mathbf{k}\mathbf{j}$ & $-\mathbf{i}$\\
$\mathbf{i}\mathbf{k}$ & $-\mathbf{j}$ &
$\mathbf{j}\mathbf{k}$ & $\mathbf{i}$ &
$\mathbf{k}\mathbf{k}$ & $-1$\\
\end{tabular}

\item\textbf{Conjugation:} $q^\star = (a, -b, -c, -d)$.

\item\textbf{Norm:} $||q|| = \sqrt{q q^\star} = \sqrt{q^\star q} = \sqrt{a^2 + b^2 + c^2 + d^2}$. We have $||p q|| = ||p||\,||q||$.

\item\textbf{Inverse:} $q^{-1} = \frac{q^\star}{||q||^2}$. $q q^{-1} = q^{-1} q = 1$.

\item\textbf{Scalar and vector parts:} $q = (r, v), r \in \mathbb{R}, v \in \mathbb{R}^3$.
\begin{eqnarray*}
(r_1, v_1) + (r_2, v_2) &=& (r_1 + r_2,\; v_1 + v_2)\\
(r_1, v_1) \cdot (r_2, v_2) &=& (r_1 r_2 - v_1 \cdot v_2,\; r_1 v_2 + r_2 v_1 + v_1 \times v_2)
\end{eqnarray*}
$q^\star = -q$ iff $r = 0$. Such quaternions are called \textit{pure imaginary}.

\item\textbf{Rotations:} Let the vector $(x, y, z)$ correspond to $v = x\mathbf{i} + y\mathbf{j} + z\mathbf{k}$. Rotations correspond to unit quaternions, $||q|| = 1$, and act on vectors via $v^\prime = q v q^{-1}$ (where $q^{-1} = q^\star$ for unit quaternions). In matrix form, $q = (a,b,c,d)$:
\begin{displaymath}
\left(\begin{array}{ccc}
a^2+b^2-c^2-d^2 & 2bc-2ad         & 2bd+2ac        \\
2bc+2ad         & a^2-b^2+c^2-d^2 & 2cd-2ab        \\
2bd-2ac         & 2cd+2ab         & a^2-b^2-c^2+d^2\\
\end{array}\right)
\end{displaymath}

Rotation by angle $\alpha$ around (unit vector) $u$ corresponds to $q = \cos(\alpha/2) + \sin(\alpha/2) u$.

\item \textbf{Angular velocity: } By differentiating $q q^\star = 1$, we obtain
\begin{displaymath}
\dot{q} q^\star = -(\dot{q} q^\star)^\star
\end{displaymath}
i.e. $\dot{q} q^\star$ is pure imaginary. The vector component is:
\begin{displaymath}
\dot{q} q^\star = (0, \omega/2)
\end{displaymath}
\end{itemize}
\end{document}
