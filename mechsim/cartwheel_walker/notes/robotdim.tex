\documentclass[a4paper]{article}
\usepackage{tikz}
\title{Cartwheel character model: dimensions and DoFs}
\begin{document}
% Parameters
\pgfmathsetmacro{\footSizeX}{0.2}
\pgfmathsetmacro{\footSizeY}{0.12}
\pgfmathsetmacro{\footSizeZ}{0.05}
\pgfmathsetmacro{\legDiameter}{0.1}
\pgfmathsetmacro{\shankDiameter}{0.1}
\pgfmathsetmacro{\thighDiameter}{0.1}
\pgfmathsetmacro{\legSizeZ}{1.0}
\pgfmathsetmacro{\kneeRelativePosZ}{0.5}
\pgfmathsetmacro{\legRelativeAnchorY}{0.6}

\pgfmathsetmacro{\pelvisSizeX}{0.25}
\pgfmathsetmacro{\pelvisSizeY}{0.45}
\pgfmathsetmacro{\pelvisSizeZ}{0.3}

\pgfmathsetmacro{\pelvisDiameter}{0.45}
\pgfmathsetmacro{\pelvisRadius}{\pelvisDiameter/2.}

\pgfmathsetmacro{\shankSizeZ}{\legSizeZ * \kneeRelativePosZ}
\pgfmathsetmacro{\thighSizeZ}{\legSizeZ - \shankSizeZ}

\pgfmathsetmacro{\footPosZ}{\footSizeZ/2.}
\pgfmathsetmacro{\anklePosZ}{\footSizeZ}
\pgfmathsetmacro{\shankPosZ}{\anklePosZ + \shankSizeZ/2.}
\pgfmathsetmacro{\kneePosZ}{\anklePosZ + \legSizeZ * \kneeRelativePosZ}
\pgfmathsetmacro{\thighPosZ}{\kneePosZ + \thighSizeZ/2.}
\pgfmathsetmacro{\hipPosZ}{\anklePosZ + \legSizeZ}
\pgfmathsetmacro{\pelvisPosZ}{\hipPosZ + \pelvisSizeZ/2.}

\pgfmathsetmacro{\legPosYL}{\pelvisSizeY/2.*\legRelativeAnchorY}
\pgfmathsetmacro{\legPosYR}{-\pelvisSizeY/2.*\legRelativeAnchorY}

\pgfmathsetmacro{\footPosX}{\footSizeX * 0.33 - \legDiameter/2.}

\maketitle
\section{Dimensions and DoFs}
\textbf{Notes:}
\begin{itemize}
\item The coordinate system described here differs from that in the original Cartwheel paper.
\item \textit{Character frame:} $z$ axis is up, $x$ axis is character facing direction.
\item In practice, the character frame is computed from the orientation of the pelvis\footnote{Also called the torso. However, `pelvis' has the advantage that the first letter is unique for all rigid bodies involved.} (also called the \textit{root}).
\item For vectors in the character frame (like the center of mass velocity), the $x$ component is called the \textit{sagittal} component and the $y$ component is called the \textit{coronal} component.
\end{itemize}

\begin{center}
\begin{tikzpicture}[scale=5]
\pgfmathsetmacro{\isqrttwo}{1/sqrt(2)}

\newcommand{\cmpos}[2]{
    \fill (#1,#2) -- (#1-0.02,#2) arc(180:90:0.02) -- cycle;
    \fill (#1,#2) -- (#1+0.02,#2) arc(0:-90:0.02) -- cycle;
    \draw (#1,#2) circle(0.02);
}

\newcommand{\rect}[4]{
\draw (#1-#3/2, #2-#4/2) -- (#1-#3/2, #2+#4/2) -- (#1+#3/2, #2+#4/2) -- (#1+#3/2, #2-#4/2) -- cycle;
}

\newcommand{\arect}[4]{
\draw (#1-#3/2, #2-#4/2+0.01) -- (#1, #2-#4/2) -- (#1+#3/2, #2-#4/2+0.01) -- (#1+#3/2, #2+#4/2-0.01) -- (#1, #2+#4/2) -- (#1-#3/2,#2+#4/2-0.01) -- cycle;
}

\newcommand{\arrowfront}[2]{
\draw (#1,#2) circle(0.02);
\fill (#1,#2) circle(0.005);
}

\newcommand{\arrowback}[2]{
\draw (#1,#2) circle(0.02);
\draw (#1+0.02*\isqrttwo, #2+0.02*\isqrttwo) -- (#1-0.02*\isqrttwo, #2-0.02*\isqrttwo);
\draw (#1-0.02*\isqrttwo, #2+0.02*\isqrttwo) -- (#1+0.02*\isqrttwo, #2-0.02*\isqrttwo);
}

\begin{scope}[xshift=-20]

\begin{scope}[xshift=-10, yshift=-5]
\draw [->] (0.03,0) -- (0.15,0) node[anchor=west]{$x$};
\draw [->] (0,0.03) -- (0,0.15) node[anchor=south]{$z$};
\arrowback{0}{0}
\draw (-0.01,-0.01) node[anchor=north east]{$y$};
\end{scope}

\draw (-0.1,0) -- (-0.12,0) node[anchor=east]{0};
\draw (0,-0.05) -- (0,-0.07) node[anchor=north]{0};

\draw(0, \pelvisPosZ+\pelvisSizeZ/2+0.05) node[anchor=south] {\textbf{Side view}};

% Pelvis (torso)
\rect{0}{\pelvisPosZ}{\pelvisSizeX}{\pelvisSizeZ}
\cmpos{0}{\pelvisPosZ}

% Thigh (upper leg)
\arect{0}{\thighPosZ}{\thighDiameter}{\thighSizeZ}
\cmpos{0}{\thighPosZ}

% Shank (lower leg)
\arect{0}{\shankPosZ}{\shankDiameter}{\shankSizeZ}
\cmpos{0}{\shankPosZ}

% Foot
\rect{\footPosX}{\footPosZ}{\footSizeX}{\footSizeZ}
\cmpos{\footPosX}{\footPosZ}
\end{scope}

\begin{scope}[xshift=20]
\begin{scope}[xshift=-10, yshift=-5]
\draw [->] (0.03,0) -- (0.15,0) node[anchor=west]{$y$};
\draw [->] (0,0.03) -- (0,0.15) node[anchor=south]{$z$};
\arrowfront{0}{0}
\draw (-0.01,-0.01) node[anchor=north east]{$x$};
\end{scope}

\draw (0.2,0) -- (0.22,0) node[anchor=west]{0};
\draw (0,-0.05) -- (0,-0.07) node[anchor=north]{0};

\draw(0, \pelvisPosZ+\pelvisSizeZ/2+0.05) node[anchor=south] {\textbf{Front view}};

% Pelvis (torso)
\rect{0}{\pelvisPosZ}{\pelvisSizeY}{\pelvisSizeZ}
\cmpos{0}{\pelvisPosZ}

% Thigh (upper leg)
\arect{\legPosYL}{\thighPosZ}{\thighDiameter}{\thighSizeZ}
\cmpos{\legPosYL}{\thighPosZ}
\arect{\legPosYR}{\thighPosZ}{\thighDiameter}{\thighSizeZ}
\cmpos{\legPosYR}{\thighPosZ}

% Shank (lower leg)
\arect{\legPosYL}{\shankPosZ}{\shankDiameter}{\shankSizeZ}
\cmpos{\legPosYL}{\shankPosZ}
\arect{\legPosYR}{\shankPosZ}{\shankDiameter}{\shankSizeZ}
\cmpos{\legPosYR}{\shankPosZ}

% Foot
\rect{\legPosYL}{\footPosZ}{\footSizeY}{\footSizeZ}
\cmpos{\legPosYL}{\footPosZ}
\rect{\legPosYR}{\footPosZ}{\footSizeY}{\footSizeZ}
\cmpos{\legPosYR}{\footPosZ}

\draw (\legPosYL, -0.02) node[anchor=north]{L};
\draw (\legPosYR, -0.02) node[anchor=north]{R};

\end{scope}

\begin{scope}[xshift=0]
\draw (0,\pelvisPosZ) node[anchor=center]{Pelvis/torso};
\draw (0,\hipPosZ) node[anchor=center]{Hip (HZ, HY, HX)};
\draw (0,\thighPosZ) node[anchor=center]{Thigh};
\draw (0,\kneePosZ) node[anchor=center]{Knee (KY)};
\draw (0,\shankPosZ) node[anchor=center]{Shank};
\draw (0,\anklePosZ) node[anchor=center]{Ankle (AY, AX)};
\draw (0,\footPosZ) node[anchor=north]{Foot};
\end{scope}

\end{tikzpicture}
\end{center}

\begin{center}
\begin{tabular}{ll}
\hline
footSizeX & \footSizeX \\
footSizeY & \footSizeY \\
footSizeZ & \footSizeZ \\
legDiameter & \legDiameter \\
shankDiameter & \shankDiameter \\
thighDiameter & \thighDiameter \\
legSizeZ & \legSizeZ \\
kneeRelativePosZ & \kneeRelativePosZ \\
legRelativeAnchorY & \legRelativeAnchorY \\
\hline
pelvisSizeX & \pelvisSizeX \\
pelvisSizeY & \pelvisSizeY \\
pelvisSizeZ & \pelvisSizeZ \\
\hline
pelvisDiameter & \pelvisDiameter \\
pelvisRadius$^*$ & \pelvisRadius \\
\hline
shankSizeZ$^*$ & \shankSizeZ \\
thighSizeZ$^*$ & \thighSizeZ \\
\hline
footPosZ$^*$ & \footPosZ \\
anklePosZ$*$ & \anklePosZ \\
shankPosZ$^*$ & \shankPosZ \\
kneePosZ$^*$ & \kneePosZ \\
thighPosZ$^*$ & \thighPosZ \\
hipPosZ$^*$ & \hipPosZ \\
pelvisPosZ$^*$ & \pelvisPosZ \\
\hline
legPosY\_L$^*$ & \legPosYL \\
legPosY\_R$^*$ & \legPosYR \\
\hline
footPosX$^*$ & \footPosX\\
\hline
\end{tabular}

\noindent Note: Quantities marked with an asterisk (*) are computed.
\end{center}

\section{Transformation matrices}
Define
\begin{eqnarray}
R_x(\phi_x) &=& \left(\begin{array}{ccc}
1 & 0 & 0\\
0 & c_x & -s_x\\
0 & s_x &  c_x\\
\end{array}\right)\\
R_y(\phi_y) &=& \left(\begin{array}{ccc}
c_y & 0 & s_y\\
0 & 1 & 0\\
-s_y & 0 & c_y\\
\end{array}\right)\\
R_z(\phi_z) &=& \left(\begin{array}{ccc}
c_z & -s_z & 0\\
s_z & c_z & 0\\
0 & 0 & 1\\
\end{array}\right)
\end{eqnarray}
where $c_x = \cos(\phi_x)$, $s_x = \sin(\phi_x)$, etc. We then have
\begin{eqnarray}
R_y R_x &=& \left( \begin{array}{ccc}
c_y  & s_y s_x & s_y c_x\\
0    & c_x     & -s_x\\
-s_y & c_y s_x & c_y c_x\\
\end{array}\right)\\
R_z R_y R_x &=& \left( \begin{array}{ccc}
c_z c_y & c_z s_y s_x - s_z c_x & c_z s_y c_x + s_z s_x\\
s_z c_y & s_z s_y s_x + c_z c_x & s_z s_y c_x - c_z s_x\\
-s_y    & c_y s_x               & c_y c_x              \\
\end{array}\right)
\end{eqnarray}

\section{Affine transformations}
An \textit{affine transformation} is defined via
\begin{equation}
T(x) = R x + c
\end{equation}
where $R$ is a rotation matrix (i.e. $R^\top = R^{-1}$) and $c$ is a vector.

The affine transformation associated with a rigid body is the transformation that turns vectors in the \textit{body frame} into vector in the \textit{global (world) frame}. In ODE, the transformation associated with a body is returned by $R = \mathtt{dBodyGetRotation}$ and $c = \mathtt{dBodyGetPosition}$. The position of the center of mass is zero in body coordinates, with implies that $c$ is the position of the bodies center of mass in world coordinates.

For a joint joining a parent and a child body, we define the affine transformation $t$ associated with the joint via
\begin{equation}
T_\mathrm{c} = T_\mathrm{p} \cdot t
\end{equation}
which implies
\begin{equation}
t = T_\mathrm{p}^{-1} T_\mathrm{c}
\end{equation}

For the hip joint, the torso is the parent, the thigh is the child, and the associated transform is
\begin{equation}
t_\mathrm{H} = R_z(\phi_\mathrm{HZ})\,R_y(\phi_\mathrm{HY})\,R_x(\phi_\mathrm{HX})
\end{equation}

For the knee joint, the thigh is the parent, the shank is the child, and the associated transform is
\begin{equation}
t_\mathrm{K} = R_y(\phi_\mathrm{KY})
\end{equation}

For the ankle joint, the shank is the parent, the foot is the child, and the associated transform is
\begin{equation}
t_\mathrm{A} = R_y(\phi_\mathrm{AY})\,R_x(\phi_\mathrm{AX})
\end{equation}

\subsection{Angular velocities}
From $R R^\top = 1$ we get
\begin{eqnarray}
&&\dot{R} R^\top + R \dot{R}^\top = 0\\
&&\dot{R} R^\top = -(\dot{R} R^\top)^\top
\end{eqnarray}
i.e. $\dot{R} R^\top$ is skew-symmetric.
\begin{equation}
\dot{R} R^\top =: \Omega =: \left(\begin{array}{ccc}
0 & -\omega_3 & \omega_2\\
\omega_3 & 0 & -\omega_1\\
-\omega_2 & \omega_1 & 0\\
\end{array}\right)
\end{equation}

Now consider the velocity of a point with body coordinates $x^\prime$:
\begin{equation}
x = T(x^\prime) = R x^\prime + c
\end{equation}
Time derivative (note $\dot{x}^\prime = 0$):
\begin{eqnarray}
\dot{x} &=& \dot{R} x^\prime + \dot{c}\\
&=& \dot{R}R^\top(x - c) + \dot{c}\\
&=& \Omega (x - c) + \dot{c}\\
&=:& \omega \times (x - c) + \dot{c}
\end{eqnarray}

\section{Joint angular velocities}
\subsection{Hip joint}
The angular velocity difference across the hip joint, expressed in the pelvis frame, is
\begin{equation}
a_H^{(P)} = \omega_\mathrm{HX} R_z(\phi_\mathrm{HZ}) R_y(\phi_\mathrm{HY}) e_x
+ \omega_\mathrm{HY} R_z(\phi_\mathrm{HZ}) e_y
+ \omega_\mathrm{HZ} e_z
\end{equation}

We define a transformation matrix $Q$ via
\begin{equation}
\label{eqn:qdef}
a_H^{(P)} = Q \left(\begin{array}{c}
\omega_\mathrm{HX} \\ \omega_\mathrm{HY} \\ \omega_\mathrm{HZ}
\end{array}\right)
\end{equation}

We find
\begin{equation}
Q = \left(\begin{array}{ccc}
c_z c_y & -s_z & 0\\
s_z c_y & c_z  & 0\\
-s_y    & 0    & 1
\end{array}\right)
\end{equation}
with $c_z = \cos(\phi_\mathrm{HZ})$, etc., and
\begin{equation}
\det(Q) = c_y
\end{equation}
which represents the well-known fact that Euler-type angles become singular if the second transform aligns the axes of the first and the third.

The inverse of $Q$ is given by
\begin{equation}
Q^{-1} = \left(\begin{array}{ccc}
c_z/c_y & s_z/c_y & 0\\
-s_z & c_z & 0\\
s_y c_z / c_y & s_y s_z / c_y & 1
\end{array}\right)
\end{equation}
so that
\begin{equation}
\left(\begin{array}{c}
\omega_\mathrm{HX} \\ \omega_\mathrm{HY} \\ \omega_\mathrm{HZ}
\end{array}\right)
= Q^{-1} a_H^{(P)}
\end{equation}

\subsection{Knee joint}
The angular velocity difference across the knee joint, expressed in the thigh frame, is
\begin{equation}
a_K^{(T)} = \omega_\mathrm{KY} e_y
\end{equation}

\subsection{Ankle joint}
The angular velocity difference across the ankle joint, expressed in the shank frame, is
\begin{equation}
a_A^{(S)} = \omega_{AX} R_y(\phi_\mathrm{AY}) e_x + \omega_\mathrm{AY} e_y
\end{equation}
so we have
\begin{equation}
a_A^{(S)} = \left( \begin{array}{c}
c_y \omega_\mathrm{AX}\\ \omega_\mathrm{AY}\\ -s_y \omega_\mathrm{AX}
\end{array}\right)
\end{equation}
with $c_y = \cos(\phi_\mathrm{AY})$, etc.

To extract $\omega_\mathrm{AX}$, note that
\begin{equation}
c_y \cdot (c_y \omega_\mathrm{AX}) - s_y \cdot(-s_y \omega_\mathrm{AX}) = \omega_\mathrm{AX}
\end{equation}

\section{Generalized forces}
To convert joint torques to generalized forces, note that the power supplied to the system by the generalized forces is given by the scalar product between generalized forces and generalized velocities. Considering the hip joint, this means that
\begin{equation}
\left<a_H^{(P)}, T_H^{(P)}\right> \stackrel{!}{=} \omega_\mathrm{HX} u_\mathrm{HX} + \omega_\mathrm{HY} u_\mathrm{HY} + \omega_\mathrm{HZ} u_\mathrm{HZ}
\end{equation}
where $T_H^{(P)}$ is the hip torque, expressed in the pelvis frame, and $u_\mathrm{HX}, u_\mathrm{HY}, u_\mathrm{HZ}$ are the generalized forces on the hip. $T_H^{(P)}$ and $u_\mathrm{HX}, u_\mathrm{HY}, u_\mathrm{HZ}$ are related by a linear map,
\begin{equation}
T_H^{(P)} = A \left(\begin{array}{c}
u_\mathrm{HX} \\ u_\mathrm{HY} \\ u_\mathrm{HZ}
\end{array}\right)
\end{equation}
Using (\ref{eqn:qdef}) to express $a_H^{(P)}$ in terms of $\omega_\mathrm{HX}, \omega_\mathrm{HY}, \omega_\mathrm{HZ}$, we get
\begin{equation}
\left<Q
\left(\begin{array}{c}
\omega_\mathrm{HX} \\ \omega_\mathrm{HY} \\ \omega_\mathrm{HZ}
\end{array}\right), A
\left(\begin{array}{c}
u_\mathrm{HX} \\ u_\mathrm{HY} \\ u_\mathrm{HZ}
\end{array}\right)
\right> \stackrel{!}{=} \omega_\mathrm{HX} u_\mathrm{HX} + \omega_\mathrm{HY} u_\mathrm{HY} + \omega_\mathrm{HZ} u_\mathrm{HZ}
\end{equation}
which implies $Q^\top A = 1$ or $A = (Q^\top)^{-1}$. Therefore,
\begin{equation}
T_H^{(P)} = (Q^{-1})^\top \left(\begin{array}{c}
u_\mathrm{HX} \\ u_\mathrm{HY} \\ u_\mathrm{HZ}
\end{array}\right)
\end{equation}
and
\begin{equation}
\left(\begin{array}{c}
u_\mathrm{HX} \\ u_\mathrm{HY} \\ u_\mathrm{HZ}
\end{array}\right)
= Q^\top T_H^{(P)}
\end{equation}
\end{document}
