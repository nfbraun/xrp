\documentclass[a4paper]{article}
\usepackage{tikz}
\title{Cartwheel character model: dimensions and DoFs}
\begin{document}
% Parameters
\pgfmathsetmacro{\footSizeX}{0.2}
\pgfmathsetmacro{\footSizeY}{0.12}
\pgfmathsetmacro{\footSizeZ}{0.05}
\pgfmathsetmacro{\legDiameter}{0.1}
\pgfmathsetmacro{\shankDiameter}{0.1}
\pgfmathsetmacro{\thighDiameter}{0.1}
\pgfmathsetmacro{\legSizeZ}{1.0}
\pgfmathsetmacro{\kneeRelativePosZ}{0.5}
\pgfmathsetmacro{\legRelativeAnchorY}{0.6}

\pgfmathsetmacro{\pelvisSizeX}{0.25}
\pgfmathsetmacro{\pelvisSizeY}{0.45}
\pgfmathsetmacro{\pelvisSizeZ}{0.3}

\pgfmathsetmacro{\pelvisDiameter}{0.45}
\pgfmathsetmacro{\pelvisRadius}{\pelvisDiameter/2.}

\pgfmathsetmacro{\shankSizeZ}{\legSizeZ * \kneeRelativePosZ}
\pgfmathsetmacro{\thighSizeZ}{\legSizeZ - \shankSizeZ}

\pgfmathsetmacro{\footPosZ}{\footSizeZ/2.}
\pgfmathsetmacro{\anklePosZ}{\footSizeZ}
\pgfmathsetmacro{\shankPosZ}{\anklePosZ + \shankSizeZ/2.}
\pgfmathsetmacro{\kneePosZ}{\anklePosZ + \legSizeZ * \kneeRelativePosZ}
\pgfmathsetmacro{\thighPosZ}{\kneePosZ + \thighSizeZ/2.}
\pgfmathsetmacro{\hipPosZ}{\anklePosZ + \legSizeZ}
\pgfmathsetmacro{\pelvisPosZ}{\hipPosZ + \pelvisSizeZ/2.}

\pgfmathsetmacro{\legPosYL}{\pelvisSizeY/2.*\legRelativeAnchorY}
\pgfmathsetmacro{\legPosYR}{-\pelvisSizeY/2.*\legRelativeAnchorY}

\maketitle
\textbf{Notes:}
\begin{itemize}
\item The coordinate system described here differs from that in the original Cartwheel paper.
\item \textit{Character frame:} $z$ axis is up, $x$ axis is character facing direction.
\item In practice, the character frame is computed from the orientation of the pelvis\footnote{Also called the torso. However, `pelvis' has the advantage that the first letter is unique for all rigid bodies involved.} (also called the \textit{root}).
\item For vectors in the character frame (like the center of mass velocity), the $x$ component is called the \textit{sagittal} component and the $y$ component is called the \textit{coronal} component.
\end{itemize}

\begin{center}
\begin{tikzpicture}[scale=5]
\pgfmathsetmacro{\isqrttwo}{1/sqrt(2)}

\newcommand{\cmpos}[2]{
    \fill (#1,#2) -- (#1-0.02,#2) arc(180:90:0.02) -- cycle;
    \fill (#1,#2) -- (#1+0.02,#2) arc(0:-90:0.02) -- cycle;
    \draw (#1,#2) circle(0.02);
}

\newcommand{\rect}[4]{
\draw (#1-#3/2, #2-#4/2) -- (#1-#3/2, #2+#4/2) -- (#1+#3/2, #2+#4/2) -- (#1+#3/2, #2-#4/2) -- cycle;
}

\newcommand{\arect}[4]{
\draw (#1-#3/2, #2-#4/2+0.01) -- (#1, #2-#4/2) -- (#1+#3/2, #2-#4/2+0.01) -- (#1+#3/2, #2+#4/2-0.01) -- (#1, #2+#4/2) -- (#1-#3/2,#2+#4/2-0.01) -- cycle;
}

\newcommand{\arrowfront}[2]{
\draw (#1,#2) circle(0.02);
\fill (#1,#2) circle(0.005);
}

\newcommand{\arrowback}[2]{
\draw (#1,#2) circle(0.02);
\draw (#1+0.02*\isqrttwo, #2+0.02*\isqrttwo) -- (#1-0.02*\isqrttwo, #2-0.02*\isqrttwo);
\draw (#1-0.02*\isqrttwo, #2+0.02*\isqrttwo) -- (#1+0.02*\isqrttwo, #2-0.02*\isqrttwo);
}

\begin{scope}[xshift=-20]

\begin{scope}[xshift=-10, yshift=-5]
\draw [->] (0.03,0) -- (0.15,0) node[anchor=west]{$x$};
\draw [->] (0,0.03) -- (0,0.15) node[anchor=south]{$z$};
\arrowback{0}{0}
\draw (-0.01,-0.01) node[anchor=north east]{$y$};
\end{scope}

\draw (-0.1,0) -- (-0.12,0) node[anchor=east]{0};
\draw (0,-0.05) -- (0,-0.07) node[anchor=north]{0};

\draw(0, \pelvisPosZ+\pelvisSizeZ/2+0.05) node[anchor=south] {\textbf{Side view}};

% Pelvis (torso)
\rect{0}{\pelvisPosZ}{\pelvisSizeX}{\pelvisSizeZ}
\cmpos{0}{\pelvisPosZ}

% Thigh (upper leg)
\arect{0}{\thighPosZ}{\thighDiameter}{\thighSizeZ}
\cmpos{0}{\thighPosZ}

% Shank (lower leg)
\arect{0}{\shankPosZ}{\shankDiameter}{\shankSizeZ}
\cmpos{0}{\shankPosZ}

% Foot
\rect{0.016}{\footPosZ}{\footSizeX}{\footSizeZ}
\cmpos{0.016}{\footPosZ}
\end{scope}

\begin{scope}[xshift=20]
\begin{scope}[xshift=-10, yshift=-5]
\draw [->] (0.03,0) -- (0.15,0) node[anchor=west]{$y$};
\draw [->] (0,0.03) -- (0,0.15) node[anchor=south]{$z$};
\arrowfront{0}{0}
\draw (-0.01,-0.01) node[anchor=north east]{$x$};
\end{scope}

\draw (0.2,0) -- (0.22,0) node[anchor=west]{0};
\draw (0,-0.05) -- (0,-0.07) node[anchor=north]{0};

\draw(0, \pelvisPosZ+\pelvisSizeZ/2+0.05) node[anchor=south] {\textbf{Front view}};

% Pelvis (torso)
\rect{0}{\pelvisPosZ}{\pelvisSizeY}{\pelvisSizeZ}
\cmpos{0}{\pelvisPosZ}

% Thigh (upper leg)
\arect{\legPosYL}{\thighPosZ}{\thighDiameter}{\thighSizeZ}
\cmpos{\legPosYL}{\thighPosZ}
\arect{\legPosYR}{\thighPosZ}{\thighDiameter}{\thighSizeZ}
\cmpos{\legPosYR}{\thighPosZ}

% Shank (lower leg)
\arect{\legPosYL}{\shankPosZ}{\shankDiameter}{\shankSizeZ}
\cmpos{\legPosYL}{\shankPosZ}
\arect{\legPosYR}{\shankPosZ}{\shankDiameter}{\shankSizeZ}
\cmpos{\legPosYR}{\shankPosZ}

% Foot
\rect{\legPosYL}{\footPosZ}{\footSizeY}{\footSizeZ}
\cmpos{\legPosYL}{\footPosZ}
\rect{\legPosYR}{\footPosZ}{\footSizeY}{\footSizeZ}
\cmpos{\legPosYR}{\footPosZ}

\draw (\legPosYL, -0.02) node[anchor=north]{L};
\draw (\legPosYR, -0.02) node[anchor=north]{R};

\end{scope}

\begin{scope}[xshift=0]
\draw (0,\pelvisPosZ) node[anchor=center]{Pelvis/torso};
\draw (0,\hipPosZ) node[anchor=center]{Hip};
\draw (0,\thighPosZ) node[anchor=center]{Thigh};
\draw (0,\kneePosZ) node[anchor=center]{Knee};
\draw (0,\shankPosZ) node[anchor=center]{Shank};
\draw (0,\anklePosZ) node[anchor=center]{Ankle};
\draw (0,\footPosZ) node[anchor=north]{Foot};
\end{scope}

\end{tikzpicture}
\end{center}

\begin{center}
\begin{tabular}{ll}
\hline
footSizeX & \footSizeX \\
footSizeY & \footSizeY \\
footSizeZ & \footSizeZ \\
legDiameter & \legDiameter \\
shankDiameter & \shankDiameter \\
thighDiameter & \thighDiameter \\
legSizeZ & \legSizeZ \\
kneeRelativePosZ & \kneeRelativePosZ \\
legRelativeAnchorY & \legRelativeAnchorY \\
\hline
pelvisSizeX & \pelvisSizeX \\
pelvisSizeY & \pelvisSizeY \\
pelvisSizeZ & \pelvisSizeZ \\
\hline
pelvisDiameter & \pelvisDiameter \\
pelvisRadius$^*$ & \pelvisRadius \\
\hline
shankSizeZ$^*$ & \shankSizeZ \\
thighSizeZ$^*$ & \thighSizeZ \\
\hline
footPosZ$^*$ & \footPosZ \\
anklePosZ$*$ & \anklePosZ \\
shankPosZ$^*$ & \shankPosZ \\
kneePosZ$^*$ & \kneePosZ \\
thighPosZ$^*$ & \thighPosZ \\
hipPosZ$^*$ & \hipPosZ \\
pelvisPosZ$^*$ & \pelvisPosZ \\
\hline
legPosY\_L$^*$ & \legPosYL \\
legPosY\_R$^*$ & \legPosYR \\
\hline
\end{tabular}

\noindent Note: Quantities marked with an asterisk (*) are computed.
\end{center}
\end{document}
