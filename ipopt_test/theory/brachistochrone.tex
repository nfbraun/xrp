\documentclass[a4paper]{article}
\begin{document}
\title{The Brachistochrone problem}
\author{Norbert Braun}
\maketitle
The problem is to find a curve, $y(x)$, with the constraints
\begin{equation}
y(0) = 0\mbox{ and }y(X) = H
\end{equation}
such that the time for a free particle, constrained to move on the curve under only the influence of gravity, to travel from $(0,0)$ to $(X,H)$ becomes minimal. Note that $y(x) > 0$ for all $x \in (0,X)$, otherwise, the particle will never reach the end of the curve.

\begin{equation}
E_\mathrm{pot} = -mgy
\end{equation}
\begin{eqnarray}
E_\mathrm{kin} &=& \frac{1}{2} m v^2 \\
&=& \frac{1}{2} m \left( \left(\frac{\mathrm{d}x}{\mathrm{d}t}\right)^2
+ \left(\frac{\mathrm{d}y}{\mathrm{d}x}\frac{\mathrm{d}x}{\mathrm{d}t}\right)^2 \right)\\
&=& \frac{1}{2} m \left(\frac{\mathrm{d}x}{\mathrm{d}t}\right)^2
\left(1 + \left(\frac{\mathrm{d}y}{\mathrm{d}x}\right)^2\right)
\end{eqnarray}
Conservation of energy:
\begin{equation}
E_\mathrm{kin} + E_\mathrm{pot} = 0
\end{equation}
\begin{equation}
\frac{1}{2} m \left(\frac{\mathrm{d}x}{\mathrm{d}t}\right)^2
\left(1 + \left(\frac{\mathrm{d}y}{\mathrm{d}x}\right)^2\right) - mgy = 0
\end{equation}
\begin{equation}
\left(\frac{\mathrm{d}x}{\mathrm{d}t}\right)^2
\left(1 + \left(\frac{\mathrm{d}y}{\mathrm{d}x}\right)^2\right) = 2gy
\end{equation}
\begin{equation}
\frac{\mathrm{d}x}{\mathrm{d}t} = \sqrt{\frac{2gy}{1+\left(\frac{\mathrm{d}y}{\mathrm{d}x}\right)^2}}
\end{equation}
\begin{equation}
T = \int_0^X \mathrm{d}x\;\frac{\mathrm{d}t}{\mathrm{d}x}
= \int_0^X \mathrm{d}x\;\sqrt{\frac{1+\left(\frac{\mathrm{d}y}{\mathrm{d}x}\right)^2}{2gy}}
\end{equation}
Consider line segment from $(x_-, y_-)$ to $(x_+, y_+)$,
\begin{equation}
y(x) = y_- + \frac{y_+ - y_-}{x_+ - x_-}\cdot(x-x_-)
=: y_- + a(x - x_-)
\end{equation}
Time for line segment:
\begin{eqnarray}
t(y_-, y_+) &=& \int_{x_-}^{x_+} \mathrm{d}x\; \frac{\mathrm{d}t}{\mathrm{d}x}\\
&=& \int_{x_-}^{x_+} \mathrm{d}x\;\sqrt{\frac{1+a^2}{2g(y_- + a(x - x_-))}}\\
&=& \sqrt{\frac{1+a^2}{2g}} \int_{x_-}^{x_+} \mathrm{d}x\; \sqrt{\frac{1}{y_- + a(x - x_-)}}\\
&=& \sqrt{\frac{1+a^2}{2g}} \frac{2}{a} \left[\sqrt{y_- + a(x - x_-)}\right]_{x_-}^{x_+}\\
&=& 2\sqrt{\frac{1+a^2}{2g}} \frac{\sqrt{y_+} - \sqrt{y_-}}{a}
\end{eqnarray}
This is ill-defined for $y_- = y_+$, so we rewrite it a little:
\begin{eqnarray}
t(y_-, y_+) &=& 2\sqrt{\frac{1+a^2}{2g}} \frac{\sqrt{y_+} - \sqrt{y_-}}{a} \frac{\sqrt{y_+} + \sqrt{y_-}}{\sqrt{y_+} + \sqrt{y_-}}\\
&=& \sqrt{\frac{1+a^2}{2g}} \frac{y_+ - y_-}{a} \frac{2}{\sqrt{y_+} + \sqrt{y_-}}\\
&=& \sqrt{\frac{1+a^2}{2g}} \frac{2}{\sqrt{y_+} + \sqrt{y_-}} (x_+ - x_-)\\
&=& \frac{\sqrt{(x_+ - x_-)^2 + (y_+ - y_-)^2}}{\sqrt{2 g}(\sqrt{y_+} + \sqrt{y_-})/2}
\end{eqnarray}
This is the time to travel the length of the line segment with a velocity that is the average of the velocity $v_- = \sqrt{2 g y_-}$ at the beginning and the velocity $v_+ = \sqrt{2 g y_+}$ at the end. Note the symmetry between $y_-$ and $y_+$, i.e.
\begin{equation}
t(y_-, y_+) = t(y_+, y_-)
\end{equation}

Now, let the curve $y(x)$ be approximated by $N+1$ line segments. Let $x_+ - x_- =: \Delta$ be equal for all line segments. The time $T$ is then given by
\begin{equation}
T = \sum_{k=0}^{N} t(y_k, y_{k+1})
\end{equation}
where $t$ is the time for a line segment, as above. The constraints force $y_0 = 0$ and $y_{N+1} = H$, so the $N$ free variables are $y_1, \ldots, y_N$. Thus
\begin{equation}
T = t(0, y_1) + \sum_{k=1}^{N-1} t(y_k, y_{k+1}) + t(y_N, H)
\end{equation}

We need to calculate the first and second derivatives of $T$ with respect to the free variables $y_1, \ldots, y_N$. They are given by
\begin{eqnarray}
\frac{\partial T}{\partial y_i} &=&
\left. \frac{\partial t}{\partial y_+} \right|_{(y_{i-1}, y_i)}
+ \left. \frac{\partial t}{\partial y_-} \right|_{(y_i, y_{i+1})}\\
\frac{\partial^2 T}{\partial y_i^2} &=&
\left. \frac{\partial^2 t}{\partial y_+^2} \right|_{(y_{i-1}, y_i)}
+ \left. \frac{\partial^2 t}{\partial y_-^2} \right|_{(y_i, y_{i+1})}\\
\frac{\partial^2 T}{\partial y_i \partial y_{i-1}} &=&
\left. \frac{\partial^2 t}{\partial y_+ \partial y_-} \right|_{(y_{i-1}, y_i)}\\
\frac{\partial^2 T}{\partial y_i \partial y_{i+1}} &=&
\left. \frac{\partial^2 t}{\partial y_- \partial y_+} \right|_{(y_i, y_{i+1})}\\
\frac{\partial^2 T}{\partial y_i \partial y_j} &=&
0 \; \mbox{, otherwise}
\end{eqnarray}
The sparsity structure of the $N \times N$ Hessian is thus
\begin{equation}
\left(\frac{\partial^2 t}{\partial y_i \partial y_j}\right) =
\left(\begin{array}{*{10}{c}}
  * & * & 0 & 0 & 0 & \ldots & 0 & 0 & 0 & 0\\
{}* & * & * & 0 & 0 & \ldots & 0 & 0 & 0 & 0\\
  0 & * & * & * & 0 & \ldots & 0 & 0 & 0 & 0\\
  0 & 0 & * & * & * & \ldots & 0 & 0 & 0 & 0\\
  0 & 0 & 0 & * & * & \ldots & 0 & 0 & 0 & 0\\
\vdots & \vdots & \vdots & \vdots & \vdots & \ddots & \vdots
& \vdots & \vdots & \vdots\\
  0 & 0 & 0 & 0 & 0 & \ldots & * & * & 0 & 0\\
  0 & 0 & 0 & 0 & 0 & \ldots & * & * & * & 0\\
  0 & 0 & 0 & 0 & 0 & \ldots & 0 & * & * & *\\
  0 & 0 & 0 & 0 & 0 & \ldots & 0 & 0 & * & *\\
\end{array}\right)
\end{equation}

To calculate the derivatives of $t$, we first define
\begin{equation}
l := l(y_-, y_+) := \sqrt{(x_+ - x_-)^2 + (y_+ - y_-)^2}
\end{equation}
and
\begin{equation}
v := v(y_-, y_+) := \sqrt{2g} (\sqrt{y_+} + \sqrt{y_-})/2
\end{equation}
so that
\begin{equation}
t(y_-, y_+) = \frac{l(y_-, y_+)}{v(y_-, y_+)}
\end{equation}
$l$ and $v$ are again both symmetric in $y_-$ and $y_+$. Their derivatives are
\begin{equation}
\frac{\partial l}{\partial y_+} = \frac{y_+ - y_-}{l}
\end{equation}
and
\begin{equation}
\frac{\partial v}{\partial y_+} = \frac{\sqrt{2g}}{4} \frac{1}{\sqrt{y_+}}
\end{equation}

From this, we find the derivatives of $t$
\begin{eqnarray}
\frac{\partial t}{\partial y_+}
&=& \frac{y_+ - y_-}{vl} - \frac{1}{4} \frac{\sqrt{2g}}{\sqrt{y_+}} \frac{l}{v^2}\\
\frac{\partial^2 t}{\partial y_+^2} &=&
\frac{1}{vl} - \left(\frac{y_+ - y_-}{v l^3} + 
    \frac{1}{4}\frac{\sqrt{2g}}{\sqrt{y_+}}\frac{1}{v^2l}\right)(y_+ - y_-)\\
\nonumber&&- \frac{1}{4} \sqrt{2g} \left(- \frac{l}{2v^2(\sqrt{y_+})^3} +
    \frac{1}{\sqrt{y_+}}\left(\frac{y_+ - y_-}{l v^2} - \frac{1}{2 v^3}
    \sqrt{2g}\frac{l}{\sqrt{y_+}}\right)\right)\\
\frac{\partial t}{\partial y_+ \partial y_-} &=&
-\frac{1}{vl} - \left(\frac{y_- - y_+}{v l^3} +
    \frac{1}{4} \frac{\sqrt{2g}}{\sqrt{y_-}}\frac{1}{v^2 l}\right)
    (y_+ - y_-)\\
\nonumber&& - \frac{1}{4}\frac{\sqrt{2g}}{\sqrt{y_+}}
    \left(\frac{y_- - y_+}{l v^2} - \frac{1}{2 v^3} \sqrt{2g}
    \frac{l}{\sqrt{y_-}}\right)
\end{eqnarray}
The remaining derivatives follow from symmetry:
\begin{equation}
\frac{\partial t}{\partial y_-} = \frac{\partial t}{\partial y_+}
\mbox{ and }
\frac{\partial^2 t}{\partial y_-^2} = \frac{\partial^2 t}{\partial y_+^2}
\end{equation}
\end{document}
